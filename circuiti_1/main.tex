\documentclass[a4paper]{article}

\usepackage[utf8]{inputenc}
\usepackage[T1]{fontenc}
\usepackage{textcomp}
\usepackage[italian]{babel}
\usepackage{amsmath, amssymb}
\usepackage{siunitx}
\usepackage{caption}
%dark mode
\usepackage{darkmode}


% figure support
\usepackage{import}
\usepackage{xifthen}
\pdfminorversion=7
\usepackage{pdfpages}
\usepackage{transparent}
\newcommand{\incfig}[1]{%
    \def\svgwidth{\columnwidth}
    \import{./figures/}{#1.pdf_tex}
}

\pdfsuppresswarningpagegroup=1

\begin{document}

\section{Introduzione}
Il primo insieme di esperienze volge a verificare, tramite l'utilizzo di circuiti a
corrente contuinua, fenomeni elettromagnetici. Più precisamente le misure effettuate sono mirate a:
\begin{itemize}
	\item valutare le caratteristiche degli strumenti di misura e verificare la legge di Ohm
	\item realizzare un partitore resistivo e studiarne il funzionamento
	\item misurare la caratteristica tensione-corrente di un diodo
	\item osservare gli effetti del campo magnetico generato da una spira percorsa da corrente

\end{itemize}
\section{Legge di Ohm}
\subsection{Obiettivo}
Misurare la relazione corrente-tensione ai capi di un resistore, utilizzando due diverse configurazioni del circuito per l'inserimento degli strumenti di misura
(voltometro e amperometro). A partire dai dati raccolti e dal loro confronto con quanto previsto dalla legge di Ohm, valutare come la non idealità dei multimetri incida sulle misure.
Considerare poi anche resistenze composite: in serie o in parallelo.
\subsection{Metodo}
Abbiamo innanzitutto costruito il circuito utilizzato per le misurazioni: abbiamo collegato l'alimentatore da banco, il quale fungeva da generatore di tensione, alla breadboard, tramite connettori a banana,
e la breadboard alla cassetta di resistenze mediante due cavi semplici, così da poter variare agevolmente la resistenza inserita nel circuito.
Per poter poi quantificare la relazione corrente-tensione abbiamo inserito nel circuito il voltometro, rappresentato dal multimetro palmare, e l'amperometro, rappresentata dal multimetro da banco.
Il collegamento con gli strumenti di misura è stato effettuato utilizzando due diverse configurazioni. Nella configurazione (1) il voltometro è stato collegato in parallelo con la resistenza;
nella configurazione (2) in parallelo con il generatore. 
%mettere qui immagine configurazioni con loro numerino e didascalia 'configurazioni di misura di una resistenza'
%creare collegamento numero-immagine così che possiamo richiamare configuarzione 1/2 anche dopo senza specificare cosa significano
Per entrambe le configurazioni abbiamo raccolto almeno dieci misure della coppia corrente-tensione per due diversi resistori, uno con resistenza piccola (10 Ohm) e uno con resistenza grande (1MOhm).
Nell'utilizzare il carico resistivo più basso è stato possibile variare la tensione soltanto in un range ristretto di valori, 
al fine di non raggiungere valori elevati di V così da evitare il passaggio di correnti troppo intense attraverso l'alimentatore o il multimetro da banco.
\subsection{Analisi dati}


%resistenza attesa 1 mega ohm, con errore 1\%
%cofigurazione 2 (quella giusta con voltometro in parallelo al generatore)

\centering
\begin{tabular}{|c|c|c|c|c|}
	\hline
	$R_{\text{atteso}}$            & $R_1$                           & $P_1$ & $R_2$ & $P_2$ \\
	\hline
	$1000 \pm 10 \si{\kilo\ohm}$   & $908.5 \pm 0.95 \si{\kilo\ohm}$ & $1$   &
	$999.2 \pm 1.1 \si{\kilo\ohm}$ & $0.061$                                                 \\
    \hline
	$10 \pm 0.1 \si{\ohm}$          &                                 &       &       &       \\
	\hline
\end{tabular}
\captionof{table}{Verifica legge di Ohm}
\label{tab:Verfica legge di Ohm}






\subsection{}

\end{document}
