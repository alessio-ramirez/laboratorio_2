\documentclass[a4paper]{article}

\usepackage[utf8]{inputenc}
\usepackage[T1]{fontenc}
\usepackage{textcomp}
\usepackage[italian]{babel}
\usepackage{amsmath, amssymb}
\usepackage{siunitx}
\usepackage{caption}
%dark mode
\usepackage{darkmode}


% figure support
\usepackage{import}
\usepackage{xifthen}
\pdfminorversion=7
\usepackage{pdfpages}
\usepackage{transparent}
\newcommand{\incfig}[1]{%
    \def\svgwidth{\columnwidth}
    \import{./figures/}{#1.pdf_tex}
}

\pdfsuppresswarningpagegroup=1

\begin{document}

\section{introduzione}
Il primo insieme di esperienze volge a verificare, tramite l'utilizzo di circuiti a
corrente contuinua, fenomeni elettromagnetici. Più precisamente le misure effettuate sono mirate a:
\begin{itemize}
	\item valutare le caratteristiche degli strumenti di misura e verificare la legge di Ohm
	\item realizzare un partitore e studiarne il funzionamento
	\item misurare la caratteristica tensione-corrente di un diodo
	\item osservare gli effetti del campo magnetico generato da una spira percorsa da corrente

\end{itemize}
\section{legge di ohm}

%resistenza attesa 1 mega ohm, con errore 1\%
%cofigurazione 2 (quella giusta con voltometro in parallelo al generatore)

\centering
\begin{tabular}{|c|c|c|c|c|}
	\hline
	$R_{\text{atteso}}$            & $R_1$                           & $P_1$ & $R_2$ & $P_2$ \\
	\hline
	$1000 \pm 10 \si{\kilo\ohm}$   & $908.5 \pm 0.95 \si{\kilo\ohm}$ & $1$   &
	$999.2 \pm 1.1 \si{\kilo\ohm}$ & $0.061$                                                 \\
    \hline
	$10 \pm 0.1 \si{\ohm}$          &                                 &       &       &       \\
	\hline
\end{tabular}
\captionof{table}{Verifica legge di Ohm}
\label{tab:Verfica legge di Ohm}






\subsection{}

\end{document}
